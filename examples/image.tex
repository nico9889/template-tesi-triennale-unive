\chapter{Immagini}
Scrivendo la tesi e inserendo immagini ti accorgerai che LaTeX tendenzialmente le mette nel primo posto libero che capita, spostandole anche a kilometri di distanza rispetto a dove sarebbero contestualmente utili.

Nel file .cls è definita una (non brillantemente scritta) macro utile a far si che le immagini rimangano al loro posto, sacrificando piuttosto il posizionamento del testo:

\imageLabel
{img/logo.pdf}
{Esempio di posizionamento immagine}
{0.5}
{immagine-di-esempio}

La macro è così composta:
\begin{itemize}
    \item \textbf{\textbackslash imageLabel} invoca la macro 
    \item \textbf{{img/logo.pdf}} è il percorso del file immagine (o PDF)
    \item \textbf{{Esempio di posizionamento immagine}} è la didascalia dell'immagine 
    \item \textbf{{0.5}} è la scala dell'immagine, in questo caso specifico viene ridotta la dimensione a metà
    \item \textbf{{immagine-di-esempio}} è il tag dell'immagine, molto importante per fare riferimenti all'immagine (vedi Figura \ref{fig:immagine-di-esempio}).
\end{itemize}