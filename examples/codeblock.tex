\chapter{Code Block}

Per creare blocchi di codice esteticamente molto fighi ma decisamente poco funzionali (copia-incollarli è tragico, provare per credere) puoi usare la macro definita nel .cls 

\begin{codeblock}{cpp}{main.cpp}
#include <iostream>
use namespace std;

int main{
    cout << "Hello world! \n";
    return 0;
}
\end{codeblock}

dove:
\begin{itemize}
    \item \textbf{\textbackslash begin\{codeblock\}} "invoca" la macro
    \item \textbf{\{cpp\}} è il primo argomento della macro e indica il linguaggio da usare per l'highlight
    \item \textbf{\{main.cpp\}} è il titolo del box
\end{itemize}

mentre tutto ciò che è contenuto all'interno viene considerato come codice.

Nel caso il codice fosse talmente lungo da uscire dalla pagina è possibile usare \textbf{[breakable=true]} nel seguente modo

\begin{codeblock}[breakable=true]{latex}{Come dividere un codice molto lungo}
    \begin{codeblock}[breakable=true]{cpp}{Codice C++}
        int main{
            cout << "Hello world! \n";
            return 0;
        }
    \end{codeblocK} % Attenzione al typo! Falliva il parsing riutilizzando "codeblock" perché chiude il blocco anticipatamente
\end{codeblock}

e ottenere quindi il seguente (moderatamente orrendo, ma c'è poco da fare) effetto.

Nella maggior parte dei casi questo aiuta LaTeX a impaginare in modo migliore il testo, in quanto gli è più facile spostare e incastrare i blocchi di codice dove prima non ci stavano, al costo però della leggibilità del codice.